%% start of file `template-zh.tex'.
%% Copyright 2006-2013 Xavier Danaux (xdanaux@gmail.com).
%
% This work may be distributed and/or modified under the
% conditions of the LaTeX Project Public License version 1.3c,
% available at http://www.latex-project.org/lppl/.


\documentclass[10pt,a4paper,sans]{moderncv}   % possible options include font size ('10pt', '11pt' and '12pt'), paper size ('a4paper', 'letterpaper', 'a5paper', 'legalpaper', 'executivepaper' and 'landscape') and font family ('sans' and 'roman')

% moderncv 主题
\moderncvstyle{classic}                        % 选项参数是 ‘casual’, ‘classic’, ‘oldstyle’ 和 ’banking’
\moderncvcolor{blue}                          % 选项参数是 ‘blue’ (默认)、‘orange’、‘green’、‘red’、‘purple’ 和 ‘grey’
%\nopagenumbers{}                             % 消除注释以取消自动页码生成功能

% 字符编码
% \usepackage[utf8]{inputenc}                   % 替换你正在使用的编码
% \usepackage{CJKutf8}
% \usepackage{fontspec}
% \usepackage{xunicode}
% \usepackage{xeCJK}
% 
% \setmainfont{WenQuanYi Micro Hei Light}
% \setsansfont{WenQuanYi Micro Hei Light}
% \setmonofont{WenQuanYi Micro Hei Light}
% \setCJKmainfont{WenQuanYi Micro Hei Light}
% \setCJKsansfont{WenQuanYi Micro Hei Light}
% \setCJKmonofont{WenQuanYi Micro Hei Light}


% 调整页面出血
\usepackage[scale=0.82]{geometry}
% %\setlength{\hintscolumnwidth}{3cm}           % 如果你希望改变日期栏的宽度

% moderntimeline
\usepackage[firstyear=2007,lastyear=2015]{moderntimeline}

% moderntimeline 居中
\makeatletter
\tikzset{
    tl@startyear/.append style={
        xshift=(0.5-\tl@startfraction)*\hintscolumnwidth,
        anchor=base
    }
}
\makeatother

%避免cventry加粗的bug
\renewcommand*{\cventry}[7][.25em]{%
  \cvitem[#1]{#2}{%
    {#3}%
    \ifthenelse{\equal{#4}{}}{}{, {#4}}%
    \ifthenelse{\equal{#5}{}}{}{, #5}%
    \ifthenelse{\equal{#6}{}}{}{, #6}%
    .\strut%
    \ifx&#7&%
      \else{\newline{}\begin{minipage}[t]{\linewidth}\small#7\end{minipage}}\fi}}


% 个人信息
\name{yo}{hoho}
% \title{简历题目 (可选项)}                     % 可选项、如不需要可删除本行
% % \address{街道及门牌号}{邮编及城市}            % 可选项、如不需要可删除本行
\phone[mobile]{(86) 012 3456 7890}              % 可选项、如不需要可删除本行
% \phone[fixed]{+2~(345)~678~901}               % 可选项、如不需要可删除本行
% \phone[fax]{+3~(456)~789~012}                 % 可选项、如不需要可删除本行
\email{xi.geng@outlook.com}                    % 可选项、如不需要可删除本行
\homepage{http://x7hub.github.io}                  % 可选项、如不需要可删除本行
% \extrainfo{附加信息 (可选项)}                 % 可选项、如不需要可删除本行
% \photo[64pt][0.4pt]{picture}                  % ‘64pt’是图片必须压缩至的高度、‘0.4pt‘是图片边框的宽度 (如不需要可调节至0pt)、’picture‘ 是图片文件的名字;可选项、如不需要可删除本行
% \quote{引言(可选项)}                          % 可选项、如不需要可删除本行

% 显示索引号;仅用于在简历中使用了引言
%\makeatletter
%\renewcommand*{\bibliographyitemlabel}{\@biblabel{\arabic{enumiv}}}
%\makeatother

% 分类索引
%\usepackage{multibib}
%\newcites{book,misc}{{Books},{Others}}
%----------------------------------------------------------------------------------
%            内容
%----------------------------------------------------------------------------------
\begin{document}
% \begin{CJK}{UTF8}{gbsn}                       % 详情参阅CJK文件包
\maketitle


\section{Personal Information}
\hspace{1.1cm}\cvdoubleitem{Gender:}{Male}{Birthdate:}{Aug. 1989}
%\hspace{1.1cm}\cvdoubleitem{家乡:}{山东济南}{政治面貌:}{中共党员}
% \hspace{1.1cm}\cvdoubleitem{学校:}{北京邮电大学}{学历:}{工学硕士}


\section{Education}
% \cventry{年 -- 年}{学位}{院校}{城市}{\textit{成绩}}{说明}  % 第3到第6编码可留白
% \cventry{年 -- 年}{学位}{院校}{城市}{\textit{成绩}}{说明}
\tllabelcventry{2008}{2012}{2008--2012}{BSc}{Beijing University of Posts and Telecommunications}{Information Engineering}{}{
\begin{itemize}%
\item Main courses: Data Structure, Information Theory, Communication Theory, Computer Network
\end{itemize}}
\tllabelcventry{2012}{0}{2012--2015}{MSc}{Beijing University of Posts and Telecommunications}{Signal and Information Processing}{}{
\begin{itemize}%
\item Research Orientation: Mobile Internet
\end{itemize}}

\section{Professional Skills}
\cvitem{}{Skilled in Linux. Use Arch Linux as routine operating system}
\cvitem{}{Skilled in C, C++, Android development. Familiar with Python, PHP, MySQL, Bash, Makefile}
\cvitem{}{Frequent user of development tools such as Vim, Eclipse, svn, git. Skilled in Markdown, LaTeX}
\cvitem{}{Score 534 in CET-6}


\section{Project Experience in Campus}
\tllabelcventry{2012.2}{2013}{2012.3--2012.12}{Research on green 3G-WiFi router}{NTT DoCoMo Beijing Lab}{}{}{%
\begin{itemize}%
\item Achieve data interoperability for 3G and WiFi networks in embedded Linux systems,  so that the terminal devices with WiFi capabilities can connect to the router to enjoy 3G Internet connection, and optimize energy consumption for a variety of situations;
\item Undertake the tasks of ARM board system installation, 3G network dial-up connection, WiFi driver and route forwarding, optimizing the energy consumption for FTP and HTTP services.
% \item 熟悉Linux内核编译,掌握基本Bash脚本,熟悉Socket网络编程。
\end{itemize}}

\tllabelcventry{2013}{2013.7}{2013.1--2013.10}{Research and implementation of wireless network quality measurement platform}{NTT DoCoMo Beijing Lab}{}{}{%
\begin{itemize}%
\item Android application for mobile network speed measuring(available on domestic app stores). Test items include latency, uplink and downlink network speed, and the test results and the user network parameters are uploaded to the server database, large amount of collected data is analyzed and shown;
\item Undertake the tasks of realization of core test engine with JNI, involved in the collection of user data.
% \item 熟悉Android JNI编程、HTTP协议,了解PHP、MySQL基本操作,加强了团队合作能力。
\end{itemize}}

\tllabelcventry{2013.5}{2014.1}{2013.7--2014.1}{Development of four basic applications for dual SIM card smartphones}{China Telecom Esufing Terminal Company}{}{}{%
\begin{itemize}%
\item Calendar, contact, dialpad, message application development for telecom Android dual SIM card dual standby mobile phone. Android 4.1 source code-based secondary development , extended functionality, optimized user experience with international roaming scenario;
\item Responsible for the overall project, undertake the development  tasks of calendar, contacts and dialpad.
% \item 熟练掌握Android应用开发,熟悉原生应用源码,掌握Android系统编译。
\end{itemize}}

\tllabelcventry{2013.5}{2013.9}{2013.7--2013.11}{Research of advantage analysis and application recommendations for operator's pipeline resources in the data product}{China Mobile Group Beijing Company}{}{}{%
\begin{itemize}%
\item Analysis of the value of the pipeline operator's current situation, analysis of information, information sources and algorithms available by mobile smart phones, research of strategies to enhance the operator's pipeline value;
\item Undertake analysis of the smart phone to get information from the operator's network, involved in the research of making full use of pipeline advantage and enhancing the competitiveness of products.
\end{itemize}}

% \tllabelcventry{2013.9}{0}{2013.12--}{移动网络质量评估体系}{中兴研究院}{}{}{%
% \begin{itemize}%
% \item 工作内容 1;
% \end{itemize}}

\tllabelcventry{2014}{0}{2014.1--}{EU FP7 IRSES MobileCloud Project}{}{}{}{%
\begin{itemize}%
\item Collaborative research by universities at home and abroad on mobile cloud computing related topics, integration of traditional cloud computing in the mobile Internet environment;
\item Undertake the research of Dynamic Adaptive Streaming over HTTP (DASH), propose an improved scheme in the mobile cloud computing environment, participate in the development of demonstration with Android terminal and Linux server, publish an IEEE conference paper.
\end{itemize}}


\section{Awards}


\cvitem{}{2011-2013 Third-class, Second-class and First-class Scholarship of Beijing University of Posts and Telecommunications}
\cvitem{}{2013 Outstanding students of Key Laboratory of Universal Wireless Communications, Ministry of Education}


\section{Self Evaluation}

\cvitem{}{Love technology, willing to learn, strong sense of responsibility, with the ability to adapt quickly integrate into the team}

% \cvdoubleitem{操作系统:}{Ubuntu, Fedora, CentOS}{数据库:}{MySQL, HBase\footnotemark[1], Memcached}
% \cvdoubleitem{编程语言:}{C/C++, Java, Python, Bash, Lua}{工具链:}{GCC, GDB, Binutils, LLVM/Clang}
% \cvdoubleitem{代码工具:}{Vim, Git, SVN, CMake, Eclipse}{其他工具:}{Valgrind,SystemTap,Gperftools\footnotemark[2]}
% \cvdoubleitem{文档编写:}{LaTeX, Markdown, Word, PPT}{外语水平:}{英语~6~级(549)}

% \section{毕业论文}
% \cvitem{题目}{\emph{题目}}
% \cvitem{导师}{导师}
% \cvitem{说明}{\small 论文简介}

% \section{工作背景}
% \subsection{专业}
% \cventry{年 -- 年}{职位}{公司}{城市}{}{不超过1--2行的概况说明\newline{}%
% 工作内容:%
% \begin{itemize}%
% \item 工作内容 1;
% \item 工作内容 2、 含二级内容:
%   \begin{itemize}%
%   \item 二级内容 (a);
%   \item 二级内容 (b)、含三级内容 (不建议使用);
%     \begin{itemize}
%     \item 三级内容 i;
%     \item 三级内容 ii;
%     \item 三级内容 iii;
%     \end{itemize}
%   \item 二级内容 (c);
%   \end{itemize}
% \item 工作内容 3。
% \end{itemize}}
% \cventry{年 -- 年}{职位}{公司}{城市}{}{说明行1\newline{}说明行2}
% \subsection{其他}
% \cventry{年 -- 年}{职位}{公司}{城市}{}{说明}

% \section{语言技能}
% \cvitemwithcomment{语言 1}{水平}{评价}
% \cvitemwithcomment{语言 2}{水平}{评价}
% \cvitemwithcomment{语言 3}{水平}{评价}

% \section{计算机技能}
% \cvdoubleitem{类别 1}{XXX, YYY, ZZZ}{类别 4}{XXX, YYY, ZZZ}
% \cvdoubleitem{类别 2}{XXX, YYY, ZZZ}{类别 5}{XXX, YYY, ZZZ}
% \cvdoubleitem{类别 3}{XXX, YYY, ZZZ}{类别 6}{XXX, YYY, ZZZ}

% \section{个人兴趣}
% \cvitem{爱好 1}{\small 说明}
% \cvitem{爱好 2}{\small 说明}
% \cvitem{爱好 3}{\small 说明}

% \section{其他 1}
% \cvlistitem{项目 1}
% \cvlistitem{项目 2}
% \cvlistitem{项目 3}

% \renewcommand{\listitemsymbol}{-}             % 改变列表符号

% \section{其他 2}
% \cvlistdoubleitem{项目 1}{项目 4}
% \cvlistdoubleitem{项目 2}{项目 5\cite{book1}}
% \cvlistdoubleitem{项目 3}{}

% 来自BibTeX文件但不使用multibib包的出版物
%\renewcommand*{\bibliographyitemlabel}{\@biblabel{\arabic{enumiv}}}% BibTeX的数字标签
% \nocite{*}
% \bibliographystyle{plain}
% \bibliography{publications}                    % 'publications' 是BibTeX文件的文件名

% 来自BibTeX文件并使用multibib包的出版物
%\section{出版物}
%\nocitebook{book1,book2}
%\bibliographystylebook{plain}
%\bibliographybook{publications}               % 'publications' 是BibTeX文件的文件名
%\nocitemisc{misc1,misc2,misc3}
%\bibliographystylemisc{plain}
%\bibliographymisc{publications}               % 'publications' 是BibTeX文件的文件名


\clearpage
% \end{CJK}
\end{document}


%% 文件结尾 `template-zh.tex'.
