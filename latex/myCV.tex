%% start of file `template-zh.tex'.
%% Copyright 2006-2013 Xavier Danaux (xdanaux@gmail.com).
%
% This work may be distributed and/or modified under the
% conditions of the LaTeX Project Public License version 1.3c,
% available at http://www.latex-project.org/lppl/.


\documentclass[10pt,a4paper,sans]{moderncv}   % possible options include font size ('10pt', '11pt' and '12pt'), paper size ('a4paper', 'letterpaper', 'a5paper', 'legalpaper', 'executivepaper' and 'landscape') and font family ('sans' and 'roman')

% moderncv 主题
\moderncvstyle{classic}                        % 选项参数是 ‘casual’, ‘classic’, ‘oldstyle’ 和 ’banking’
\moderncvcolor{blue}                          % 选项参数是 ‘blue’ (默认)、‘orange’、‘green’、‘red’、‘purple’ 和 ‘grey’
%\nopagenumbers{}                             % 消除注释以取消自动页码生成功能
% \definecolor{color2}{rgb}{0,0,0}		% black

% 字符编码
% \usepackage[utf8]{inputenc}                   % 替换你正在使用的编码
% \usepackage{CJKutf8}
\usepackage{fontspec}
\usepackage{xunicode}
\usepackage[slantfont,boldfont,CJKnumber]{xeCJK}

\setmainfont{WenQuanYi Micro Hei Light}
\setsansfont{WenQuanYi Micro Hei Light}
\setmonofont{WenQuanYi Micro Hei Light}
\setCJKmainfont{WenQuanYi Micro Hei Light}
\setCJKsansfont{WenQuanYi Micro Hei Light}
\setCJKmonofont{WenQuanYi Micro Hei Light}


% 调整页面出血
\usepackage[scale=0.82]{geometry}
% %\setlength{\hintscolumnwidth}{3cm}           % 如果你希望改变日期栏的宽度

% moderntimeline
\usepackage[firstyear=2007,lastyear=2015]{moderntimeline}

% moderntimeline 居中
\makeatletter
\tikzset{
    tl@startyear/.append style={
        xshift=(0.5-\tl@startfraction)*\hintscolumnwidth,
        anchor=base
    }
}
\makeatother

%避免cventry加粗的bug,第6项右对齐
\renewcommand*{\cventry}[7][.25em]{%
  \linespread{1.2}
  \cvitem[#1]{#2}{%
    {#3}%
    \ifthenelse{\equal{#4}{}}{}{, {#4}}%
    \ifthenelse{\equal{#5}{}}{}{, #5}%
    \ifthenelse{\equal{#6}{}}{}{\hfill#6}%
%     .\strut%
    \ifx#7&%
      \else{\newline{}\begin{minipage}[t]{\linewidth}\small#7\end{minipage}}
    \fi
  }
}

\renewcommand{\labelitemi}          {\strut\textcolor{color1}{\large$\circ$}}% the \rmfamily is required to force Latin Modern fonts when using sans serif, as OMS/lmss/m/n is not defined and gets substituted by OMS/cmsy/m/n


% 个人信息
\name{哟}{嚯嚯}
% \title{简历题目 (可选项)}                     % 可选项、如不需要可删除本行
% % \address{街道及门牌号}{邮编及城市}            % 可选项、如不需要可删除本行
\phone[mobile]{012 3456 7890}              % 可选项、如不需要可删除本行
% \phone[fixed]{+2~(345)~678~901}               % 可选项、如不需要可删除本行
% \phone[fax]{+3~(456)~789~012}                 % 可选项、如不需要可删除本行
\email{x7.0@outlook.com}                    % 可选项、如不需要可删除本行
% \homepage{http://x7hub.github.io}                  % 可选项、如不需要可删除本行
% \extrainfo{附加信息 (可选项)}                 % 可选项、如不需要可删除本行
% \photo[64pt][0.4pt]{picture}                  % ‘64pt’是图片必须压缩至的高度、‘0.4pt‘是图片边框的宽度 (如不需要可调节至0pt)、’picture‘ 是图片文件的名字;可选项、如不需要可删除本行
% \quote{引言(可选项)}                          % 可选项、如不需要可删除本行

% 显示索引号;仅用于在简历中使用了引言
%\makeatletter
%\renewcommand*{\bibliographyitemlabel}{\@biblabel{\arabic{enumiv}}}
%\makeatother

% 分类索引
%\usepackage{multibib}
%\newcites{book,misc}{{Books},{Others}}
%----------------------------------------------------------------------------------
%            内容
%----------------------------------------------------------------------------------
\begin{document}
% \begin{CJK}{UTF8}{gbsn}                       % 详情参阅CJK文件包
\maketitle


% \section{个人资料}
% \hspace{1.1cm}\cvdoubleitem{性别:}{男}{出生日期:}{1989年8月}
%\hspace{1.1cm}\cvdoubleitem{家乡:}{山东济南}{}{}%{政治面貌:}{中共党员}
% \hspace{1.1cm}\cvdoubleitem{学校:}{北京邮电大学}{学历:}{工学硕士}

\vspace{-25pt}
\httplink[http://x7res.info]{x7res.info}
\vspace{20pt}

\section{教育背景}

\tllabelcventry{2012}{0}{2012--2015}{工学硕士}{北京邮电大学}{信号与信息处理}{}{
% \begin{itemize}%
% \item 研究方向: 移动互联网
% \end{itemize}
}

\vspace{-10pt}

\tllabelcventry{2008}{2012}{2008--2012}{工学学士}{北京邮电大学}{信息工程}{}{
% \begin{itemize}%
% \item 主要课程: 数据结构, 信息论, 通信原理, 计算机网络
% \end{itemize}
}

\vspace{-15pt}


\section{实习经历}

\tllabelcventry{2014.7}{0}{2014.7--}{手机微博客户端技术开发}{}{}{新浪微博}{%
\vspace{-8pt}
\begin{itemize}%
\item 参考iBeacon和Nearby,利用蓝牙、WiFi、声波等,在Android上的近距离通信技术的探索
\item 微博客户端的性能测试,开发自动测试脚本,研究提高图形性能的方案
\item Android编程~~~~~~~~\textcolor{color1}{\large$\circ$}~~Shell编程~~~~~~~~\textcolor{color1}{\large$\circ$}~~Python编程
\end{itemize}}


\section{项目经历}

\tllabelcventry{2014.5}{0}{2014.5--}{LBS商场打折信息分享平台}{}{}{参赛项目}{%
\vspace{-8pt}
\begin{itemize}%
\item 用户间互相分享打折信息的平台,包括Android应用``折扣控''和后台服务器
\item 开发,合作开发Android应用,独立开发后台服务器,使用Yii框架
\item Android编程~~~~~~~~\textcolor{color1}{\large$\circ$}~~PHP\&MySQL编程
\end{itemize}}

\tllabelcventry{2014.5}{2014.7}{2014.5--2014.7}{移动网络质量评估体系}{}{}{中兴研究院}{%
\vspace{-8pt}
\begin{itemize}%
\item 针对智能终端不同类型的业务量化QoS评价
\item 开发,承担开发网络质量参数收集平台,协助分析网络QoS体系
\item Android编程~~~~~~~~\textcolor{color1}{\large$\circ$}~~PHP\&MySQL编程%~~~~~~~~\textcolor{color1}{\large$\circ$}~~阅读文献
\end{itemize}}

\tllabelcventry{2014}{2014.5}{2014.1--2014.5}{移动云计算研究EU FP7 IRSES MobileCloud Project}{}{}{欧盟}{%
\vspace{-8pt}
\begin{itemize}%\setlength{\itemsep}{1pt }
\item 传统云计算在移动互联网环境的融合与应用
\item 理论研究和开发,承担基于HTTP的动态自适应流媒体(DASH)的研究,开发利用Android的演示系统,在IEEE会议中发表了论文
\item Android编程%~~~~~~~~\textcolor{color1}{\large$\circ$}~~阅读文献
\end{itemize}}

% \tllabelcventry{2013.5}{2013.9}{2013.7--2013.8}{运营商管道资源在数据产品中的优势分析及应用建议研究}{}{}{北京移动}{%
% \vspace{-8pt}
% \begin{itemize}%\setlength{\itemsep}{1pt }
% \item 分析运营商的管道价值现状及提升策略;
% \item 理论研究,承担智能手机利用网络信息的定位算法的分析研究。
% \item 熟悉定位原理和方法~~~~~~~~\textcolor{color1}{\large$\circ$}~~阅读文献能力
% \end{itemize}}

\tllabelcventry{2013.5}{2014.1}{2013.7--2014.1}{双卡智能手机应用开发}{}{}{电信天翼}{%
\vspace{-8pt}
\begin{itemize}%\setlength{\itemsep}{1pt }
\item 针对电信Android双卡双待手机的应用开发
\item 开发,修改Google原生应用,承担日历、通讯录、拨号盘三个模块的开发
\item Android编程
\end{itemize}}

\tllabelcventry{2013}{2013.7}{2013.1--2013.7}{无线网络质量测量平台}{}{}{NTT DoCoMo}{%
\vspace{-8pt}
\begin{itemize}%\setlength{\itemsep}{1pt }
\item Android手机网络测速软件''手机网速摇摇看'',各大应用市场可下载
\item 开发,承担核心测试引擎的JNI实现,用户数据收集
\item Android JNI编程~~~~~~~~\textcolor{color1}{\large$\circ$}~~PHP\&MySQL编程
\end{itemize}}

\tllabelcventry{2012.2}{2013}{2012.3--2012.12}{绿色3G-WiFi路由器}{}{}{NTT DoCoMo}{%
\vspace{-8pt}
\begin{itemize}%\setlength{\itemsep}{1pt }
\item 在嵌入式Linux系统中实现通过WiFi热点连接3G网络,并优化能耗
\item 开发,承担ARM软件和脚本开发,针对FTP和HTTP业务优化能耗
\item Linux操作~~~~~~~~\textcolor{color1}{\large$\circ$}~~C语言Socket编程~~~~~~~~\textcolor{color1}{\large$\circ$}~~Bash脚本~~~~~~~~\textcolor{color1}{\large$\circ$}~~内核编译
\end{itemize}}


\section{专业技能}
\cvitem{}{熟练掌握Linux基本操作,以Arch Linux为日常操作系统}
\cvitem{}{熟练掌握Android、C应用开发,熟悉PHP、Shell}
\cvitem{}{熟悉Vim、Eclipse、svn、git等开发工具,掌握Markdown、LaTeX编写文档}
\cvitem{}{大学英语六级考试534分}


% \section{获奖情况}
% 
% 
% \cvitem{}{2011-2013年 北京邮电大学三等、二等、一等奖学金}
% \cvitem{}{2013年 泛网无线通信教育部重点实验室优秀学生}


% \section{自我评价}
% 
% \cvitem{}{热爱技术,乐于学习,希望从事IT相关职业}
% \cvitem{}{熟悉软件开发,适应敏捷式迭代开发的体系,参与完成多个软件开发项目}
% \cvitem{}{工作高效,能承受压力,在一周内实现并改进一套演示系统,完成高质量论文并成功发表}


\clearpage
% \end{CJK}
\end{document}


%% 文件结尾 `template-zh.tex'.
